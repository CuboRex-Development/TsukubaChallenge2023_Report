%#!platex
% NLproceedings サンプル文書.パブリックドメイン.
\documentclass[
  platex, dvipdfmx,  % ワークフローは必ず明示的に指定する
]{rbproceedings}
%english option
%\documentclass[platex, dvipdfmx, english]{rbproceedings}
%#!uplatex
%\documentclass[uplatex,dvipdfmx]{rbproceedings}
%#!lualatex
%\documentclass[lualatex]{rbproceedings}

% パッケージ
\usepackage{graphicx,xcolor}  % グラフィックス関連
\usepackage{url}
\usepackage{pxrubrica}  % ルビ

%% option 不要な場合はコメントアウト
% \usepackage{jlreq-deluxe}  % 多書体化(otf パッケージは使用しない)、Ubuntu 22.04 以降
\usepackage{hyperref}
\hypersetup{
	colorlinks=true,
    citecolor=blue,
    linkcolor=blue,
    urlcolor=blue,
	pdfborder={0 0 0},
}
\usepackage[verb]{bxghost}    % \verb 前後に適切な和欧文間スペース

% 参考文献のフォントサイズを指定
%\renewcommand{\bibfont}{\normalsize} % 標準サイズ
%\renewcommand{\bibfont}{\footnotesize} % より小さく

% \emph をゴシックかつ太字に(比較的新しい LaTeX が必要)
\DeclareEmphSequence{\gtfamily\sffamily\bfseries}

% 著者用マクロ
\newcommand{\pkg}[1]{\textsf{#1}}
\newcommand{\code}[1]{\texttt{#1}}
\newcommand{\comment}[1]{\textcolor{red}{#1}}

\title{\pkg{NLProceedings}文書クラス サンプル文書}
% \author{%
%   言語太郎 \\ (言語大) \\ \texttt{taro@nlp.example.com}\and
%   言語花子 \\ (言語大) \\ \texttt{hanako@nlp.example.com}}

\author{%
  佐藤□□□${}^{1}$ 鈴木□□${}^{1}$ 高橋□□□${}^{2}$ 田中□□□□${}^{4}$\\
 伊藤□□□${}^{1,3,4}$ 渡辺□□□□${}^{1,3,4}$\\
${}^{1}$○○○○○○○○○○○○○大学大学院
 ${}^{2}$△△△△△△△△△大学 言語処理学部\\
${}^{3}$△△△△△△△△△△△△△△△△△△△株式会社  ${}^{4}$○○○○○○研究所\\ \texttt{\{sato,suzuki,ito\}@example1.jp}
 \texttt{takahashi@example2.jp}\\
 \texttt{\{tanaka,watanabe\}@example3.jp}}

\begin{document}

\maketitle

\begin{abstract}
NLP2022より,読者の論文理解を促進するため,所定のフォーマットの一部として投稿論文の概要を記載することにした(NLP2021までは概要は記載する必要がなく,ほぼ全ての論文で概要が存在しなかった).
分量の目安は日本語/英語ともに「8〜13行」とする.概要が8〜13行を満たさなくても賞選考対象外や不採択になることはない.ただし,極端に短い/長い概要にならないように留意すること.
日本語の場合は,文書クラスにより一行23文字に設定されているため,161文字から299文字相当になる.
\end{abstract}

\section{はじめに}
\pkg{NLProceedings}文書クラスはW3Cにより策定されている『日本語組版の要件』%
\cite{JLREQ}に準拠することを目指す\pkg{jlreq}クラスをベースにしている.
ただし,本文書クラスでは紙面スペースの都合上,多くの余白値をかなり詰めるように設定
しており,例えば行間は\ruby{外国人参政権}{がい|こく|じん|さん|せい|けん}の
ようにルビを振れる最小限の余白に設定してある.

自然言語処理分野の論文では,単純なテキストのみならず,しばしば数式
%
\begin{equation}
P(B\mid A) = \frac{P(A\mid B)P(B)}{P(A)}
\end{equation}
%
や箇条書き
%
\begin{itemize}
\item 第1の項目
\item 第2の項目
\end{itemize}
%
といった構造も用いられるが,これらもよく知られた文書クラス(例えば
\pkg{jsarticle}等)と同様のシンタックスで利用できる.

\section{図表の挿入}

図表についても通常の\LaTeX と同じ方法を用いることができる.

\subsection{図について}

図の挿入は通常\pkg{graphicx}パッケージによって行う(図\ref{fig:sample}).
クラスオプションにワークフロー(\code{dvipdfmx}等)を指定していれば,
各パッケージを読み込む際に何度も同じオプションを指定する必要はない.
%
\begin{figure}[t]
\centering
\includegraphics[width=3cm]{example-image-a}
\caption{何らかの図}
\label{fig:sample}
\end{figure}

\subsection{表について}

表組みももちろん利用できるが,図とは異なりキャプションは表本体の上に付ける
(表\ref{tab:sample}).
%
\begin{table}[t]
\centering
\caption{適当な表}
\label{tab:sample}
\begin{tabular}{llcc}
\hline
日本語 & Japanese & ほげほげ & ふげふげ \\
英語 & English & hogehoge & fugefuge \\
\hline
\end{tabular}
\end{table}

\section{参考文献}
参考文献の参照例.
\begin{itemize}
\item 論文誌の参照例 \cite{Article_01}
\item 本の参照例 \cite{Book_02}
\item 国際会議の参照例 \cite{Inproc_03}
\item 技術報告の参照例 \cite{Techrep_05}
\item Webページの参照例 \cite{Web_06}
\end{itemize}

\section{Writing in English}
This paragraph shows an English sample.
There is no problem with writing your manuscript in English.
If you write in LaTeX, please use the distributed document class with the \code{english} option:
\begin{quote}
\verb|\documentclass[|\\
\verb|  platex, dvipdfmx, english]{nlp2022}|
\end{quote}

% 参考文献
\bibliographystyle{junsrt}
\bibliography{myrefs}

\end{document}
